\documentclass[12pt]{article}
\usepackage[utf8]{inputenc}
\usepackage[leterpaper,top=0cm,bottom=1.8cm,left=1.5cm,right=1.5cm,marginparwidth=1.75cm]{geometry}
\usepackage{multicol}
\usepackage{enumitem}

\begin{document}

\renewcommand{\thesection}{\Roman{section}}

\setcounter{page}{209}

\title{\begin{center}
\date{}
\author{}
\textbf{Principles of building a system for automating
the activities of a process engineer based on an
ontological approach within the framework of
the Industry 4.0 concept} 
\end{center}}
\maketitle

\vspace{-3cm}

\begin{center}
\begin{tabular}{c c c}
\small{Valery Taberko, Dzmitry Ivaniuk}  & \small{Nikita Zotov, Maksim Orlov} & \hspace{-1cm}\small{Oleksandr Pupena, Nataliia Lutska} \\
\small{\textit{JSC "Savushkin Product"}} & \small{\textit{Belarusian State University}} & \hspace{-0.5cm}\small{\textit{National University of Food Technologies}} \\
\small{Brest, Republic of Belarus} & \small{\textit{of Informatics and Radioelectronics}} & \hspace{-1cm}\small{Kyiv, Ukraine} \\
\small{tab@pda.savushkin.by} & \small{Minsk, Republic of Belarus} & \hspace{-1cm}\small{pupena\_sun@ukr.net} \\
\small{id@pda.savushkin,by} & \small{nikita.zotov.belarus@gmail.com,} & \hspace{-1cm}\small{lutskanm2017@gmail.com} \\
& \small{orlovmassimo@gmail.com} & \\
\end{tabular}
\end{center}
\vspace{0.1cm}

\setcounter{columns}{3}

\begin{multicols}{2}
\setlength\parindent{1.5em}
\footnotesize{\textbf{\textit{Abstract}—In this article, an approach to the continuous
development of automation of the processes of creating,
developing and applying standards based on the OSTIS
Technology is proposed. Examples of these processes due
to the involvement of end-users of the system using the tools
and mechanisms of the OSTIS Technology are considered.
Examples of further formalization of standards within the
framework of the proposed approach are given.}
\par
\setlength\parindent{1.5em}
\textbf{\textit{Keywords}—automation of manufacturing processes, information service, ontological production model, Industry
4.0, ontology, knowledge base, OSTIS Technology}}

\vspace{-0.4cm}
\begin{center}
   \section{\small INTRODUCTION} 
\end{center}
\vspace{-0.40cm}

\normalsize{The implementation of the Industry 4.0 concept at
production facilities is accompanied by the development
of a single ontological production model, which is the core
of the complex information service of the enterprise. At
the first stage of developing such an enterprise model, it
is necessary to nest data on the lower level of production,
namely on the manufacturing process and equipment. As
the source of this data, P&ID-schemes of production
can serve. Thus, the formalization of the ISA 5.1 [1]
standard is necessary to work with P&ID-schemes, which
are widely used in control systems together with the ISA
88 [2] standard and allow describing the lower level of
production in full. At the same time, it is also necessary
to consider the approach of formalization of the subject
domain based on the ISO 15926 [3], [4] standard, which
describes the integration of data on the life cycle of
processing enterprises into a single ontological storage.
New users will be added: an automation engineer and a
master, who implement the new capability of the intelligent search together with the developed model. For the
current user – the operator of the manufacturing process
– the implementation of the mechanism for obtaining
intelligent information that covers both particular and
common issues of the manufacturing process, equipment,}

\vspace{0.2cm}
\begin{center}
    \section{\small BRIEFLY ABOUT ISA-5.1}
\end{center}
\vspace{-0.8cm}

\normalsize{This standard describes the rules for drawing up
functional schemes for the automation of manufacturing
processes. Such schemes allow the graphical representation of the production technology and equipment as
well as define the rules for identifying equipment and
measuring and automation tools for design and service
purposes. Figure 1 shows an example of a functional
scheme.}
\par
\setlength\parindent{1.5em}
\normalsize{The functional scheme shows: the coagulator itself (the
unit), the lines (the machine) and the valves (the control
device). Different colors indicate the purpose of the lines
(red – washing, blue – mixture, green – whey, black –
product). This fragment allows getting an insight into
which devices are used and how they are connected.
}

\vspace{-0.4cm}
\begin{center}
\section{\small ONTOLOGIES IN PRODUCTION}
\end{center}
\vspace{-0.5cm}

\normalsize{The ISA-88 article described how to use the knowledge
base on the basis of the OSTIS [7] ontologies to train
the operator with complex concepts, search for objects
according to ISA-88 and their interrelations. The need for
knowledge bases for production is not restricted to the
above. Among the most complex problems that can be
solved using knowledge bases on the basis of ontologies,
there are:}
\begin{itemize}
\item decision support in unforeseen situations as well as
start-ups and ends;
\end{itemize}
\end{multicols}
\end{document}
