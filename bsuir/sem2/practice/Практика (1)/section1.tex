%Пример

\begin{SCn}
\begin{small}

\textbf{Определение требований к платформе и ее функциональности[1]}\\
\scnrelfromset{требования и функциональность:}{Общее представление\\
\scnaddlevel{1}
    \scntext{описание}{Определение требований к платформе и ее функциональности обеспечивает ясное представление о том, какая платформа должна быть создана и какие задачи она должна выполнять. Это служит основой для дальнейшего проектирования компонентов и модулей платформы.
}
\scnaddlevel{-1}
\scnaddlevel{1}
\scnrelfromset{Требования}{Сбор требований\\
\scnaddlevel{1}
 \scntext{описание}{Взаимодействие с заинтересованными сторонами, включая пользователей, заказчиков и другие команды, чтобы понять их потребности и ожидания от платформы. Это может включать проведение собеседований, анализ существующих систем или проведение опросов.
}
\scnaddlevel{-1}
;Формулирование требований\\
\scnaddlevel{1}
    \scntext{описание}{Преобразование собранных данных в конкретные требования, которым должна соответствовать платформа. Требования могут быть функциональными (что платформа должна делать) или нефункциональными (какие ограничения и качества должна обладать платформа).
}
\scnaddlevel{-1}
;Приоритизация требований\\
\scnaddlevel{1}
    \scntext{описание}{Определение относительной важности каждого требования и их приоритетов. Это позволяет сосредоточить усилия на реализации ключевых функций и обеспечении основных требований.
}
\scnaddlevel{-1}
;Разработка сценариев использования\\
\scnaddlevel{1}
    \scntext{описание}{Создание типичных сценариев использования платформы для более полного понимания ее функциональности. Это помогает идентифицировать дополнительные требования и определить, как пользователи будут взаимодействовать с платформой.
}
\scnaddlevel{-1}
;Документирование требований\\
\scnaddlevel{1}
    \scntext{описание}{Определение относительной важности каждого требования и их приоритетов. Это позволяет сосредоточить усилия на реализации ключевых функций и обеспечении основных требований.
}
\scnaddlevel{-1}
}\scnaddlevel{-1}
\scnrelfromset{\hyperlink{p2} {Архитектура и дизайн платформы[2]}}{Выбор архитектурной модели\\
\scnaddlevel{1}
    \scntext{описание}{Изучение различных архитектурных моделей и их применимость к конкретной платформе.
Анализ требований и особенностей платформы для определения наиболее подходящей архитектуры.
Выбор между монолитной, микросервисной, распределенной или иной архитектурой в зависимости от потребностей платформы.
}
\scnaddlevel{-1}
;Определение требований к платформе:\\
\scnaddlevel{1}
    \scntext{описание}{Сбор и анализ требований, предъявляемых к платформе со стороны пользователей и системных компонентов.
Учет функциональных, производительностных, безопасностных и масштабируемых требований.
Формулировка конкретных требований, которым должна соответствовать платформа.}
\scnaddlevel{-1}
;Проектирование компонентов и модулей платформы:\\
\scnaddlevel{1}
    \scntext{описание}{Разбиение платформы на логические компоненты и модули в соответствии с требованиями и функциональностью.
Определение интерфейсов и взаимодействия между компонентами для обеспечения согласованной работы платформы.
Учет принципов разделения ответственности (separation of concerns) и модульности при проектировании компонентов.
}
\scnaddlevel{-1}
;Управление зависимостями и взаимодействием между компонентами:\\
\scnaddlevel{1}
    \scntext{описание}{Идентификация зависимостей между компонентами платформы.
Определение стратегии управления зависимостями, включая выбор инструментов и подходов.
Разработка механизмов взаимодействия и коммуникации между компонентами для обеспечения эффективной работы платформы.
}
\scnaddlevel{-1}
}}
\scnaddlevel{-1}
\scnaddlevel{1}
\scnrelfromset{\hyperlink{p3}{Инструменты разработки и автоматизации}}{Выбор инструментов разработки\\
\scnaddlevel{1}
 \scntext{описание}{Изучение различных инструментов разработки, таких как интегрированные среды разработки (IDE), редакторы кода, средства отладки и средства анализа кода. Определение, какие инструменты наилучшим образом соответствуют требованиям и потребностям разработчиков платформы.
}
\scnaddlevel{-1}
;Автоматизация процессов\\
\scnaddlevel{1}
    \scntext{описание}{Внедрение автоматизации для развертывания, тестирования и сборки платформы. Использование инструментов, таких как системы непрерывной интеграции (CI/CD), для автоматизации процессов сборки и развертывания новых версий платформы. Автоматическое тестирование позволяет обнаруживать ошибки и обеспечивать надежность и качество платформы.
}
\scnaddlevel{-1}
;Системы контроля версий\\
\scnaddlevel{1}
    \scntext{описание}{Использование систем контроля версий, таких как Git, для управления кодом платформы. Система контроля версий позволяет отслеживать изменения в коде, управлять ветвлением и слиянием кода, а также обеспечивает коллаборацию разработчиков и восстановление предыдущих версий кода при необходимости.
}
\scnaddlevel{-1}
;Конфигурационное управление\\
\scnaddlevel{1}
    \scntext{описание}{Использование инструментов конфигурационного управления, таких как Ansible, Chef или Puppet, для управления конфигурацией и развертывания компонентов платформы. Это позволяет автоматизировать установку и настройку платформы на различных окружениях и обеспечивает консистентность конфигурации.
}
\scnaddlevel{-1}
}\scnaddlevel{-1}
\textbf{\hyperlink{p4}{Организация процесса разработки платформы[4]}}\\
\scnaddlevel{1}
    \scntext{описание}{Эффективная организация процесса разработки платформы является ключевым фактором для достижения успеха. В этом пункте рассмотрим методологии разработки, распределение задач и управление командой разработчиков платформы, а также взаимодействие с другими командами и проектами в рамках организации.
}
\scnaddlevel{-1}
\scnaddlevel{1}
\scnrelfromset{Процесс}{Методологии разработки\\
\scnaddlevel{1}
 \scntext{описание}{Выбор и применение методологии разработки, наиболее подходящей для platform engineering. Это может быть Agile, Scrum, Kanban или другая методология, которая обеспечит гибкость, сотрудничество и ускоренную разработку. Применение методологии помогает управлять жизненным циклом разработки платформы, устанавливать приоритеты задач и обеспечивать прозрачность и коммуникацию в команде.
}
\scnaddlevel{-1}
;Распределение задач и управление командой разработчиков\\
\scnaddlevel{1}
    \scntext{описание}{Определение ролей и ответственностей в команде разработчиков платформы. Распределение задач, планирование и отслеживание прогресса работы. Важно установить эффективные коммуникационные каналы и обеспечить коллаборацию между членами команды. Также следует обеспечить мотивацию и поддержку разработчиков для достижения общих целей.
}
\scnaddlevel{-1}
;Взаимодействие с другими командами и проектами\\
\scnaddlevel{1}
    \scntext{описание}{становление эффективного взаимодействия с другими командами и проектами в рамках организации. Это может включать проведение регулярных совещаний, синхронизацию требований и планов, обмен опытом и знаниями. Взаимодействие помогает избежать конфликтов и обеспечивает согласованность и сотрудничество между различными командами в организации.
}
\scnaddlevel{-1}
}\scnaddlevel{-1}
\textbf{\hyperlink{p3}{Управление конфигурацией и версионирование[3]}}\\
\scnaddlevel{1}
    \scntext{описание}{Управление конфигурацией и версионирование являются важными аспектами разработки и поддержки платформы. Они позволяют эффективно управлять изменениями, обеспечивать совместимость и обратную совместимость платформы, а также контролировать конфигурацию компонентов. Ниже представлены ключевые элементы, связанные с управлением конфигурацией и версионированием платформы:
}
\scnaddlevel{-1}
\scnaddlevel{1}
\scnrelfromset{Ключевые элементы}{Управление конфигурацией платформы\\
\scnaddlevel{1}
 \scntext{описание}{Определение и управление конфигурацией компонентов платформы, включая управление изменениями, управление конфигурационными элементами и управление версиями. Это включает установление процедур и политик, которые обеспечивают консистентность и стабильность платформы.
}
\scnaddlevel{-1}
;Методы версионирования\\
\scnaddlevel{1}
    \scntext{описание}{Применение методов версионирования для идентификации и управления различными версиями платформы. Это может включать применение семантического версионирования, где номер версии состоит из основного, минорного и патч-релиза. Каждое изменение платформы должно быть явно отражено в версионировании.}
\scnaddlevel{-1}
;Управление изменениями\\
\scnaddlevel{1}
    \scntext{описание}{Установление процесса управления изменениями, который определяет, как новые функциональности, исправления ошибок или другие изменения вносятся в платформу. Это может включать оценку и утверждение изменений, использование системы отслеживания задач и контроль за процессом слияния изменений.
}
\scnaddlevel{-1}
;Обеспечение совместимости и обратной совместимости\\
\scnaddlevel{1}
    \scntext{описание}{Учет совместимости платформы с различными версиями компонентов, операционных систем, сторонних интеграций и других факторов. Обратная совместимость также является важной, чтобы новые версии платформы могли работать существующими приложениями и интеграциями, минимизируя потенциальные проблемы.
}\scnaddlevel{-1}}
\end{small}
\end{SCn}